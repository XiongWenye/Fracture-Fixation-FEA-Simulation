\documentclass{ctexart}
\usepackage{amsmath}
\usepackage{bm}

\begin{document}

\section*{常应变三角单元(CST)刚度矩阵推导}

\subsection*{1. 单元定义与位移假设}
常应变三角单元是二维有限元中最简单的单元,由三个节点组成,每个节点有2个自由度($u, v$)。单元内位移采用\textbf{线性插值}:
\[
\begin{cases}
u(x,y) = a_1 + a_2 x + a_3 y \\
v(x,y) = a_4 + a_5 x + a_6 y 
\end{cases}
\]
写成矩阵形式:
\[
\mathbf{u} = \begin{bmatrix} u \\ v \end{bmatrix} = \mathbf{N} \mathbf{d}^e
\]
其中:
\begin{itemize}
  \item $\mathbf{N}$ 为形函数矩阵,$\mathbf{N} = \begin{bmatrix} N_1 & 0 & N_2 & 0 & N_3 & 0 \\ 0 & N_1 & 0 & N_2 & 0 & N_3 \end{bmatrix}$。
  \item $\mathbf{d}^e$ 为节点位移向量,$\mathbf{d}^e = [u_1, v_1, u_2, v_2, u_3, v_3]^T$。
  \item 形函数 $N_i$ 满足 $N_i(x_j, y_j) = \delta_{ij}$,且 $\sum N_i = 1$。
\end{itemize}

\subsection*{2. 应变-位移关系}
二维弹性力学中,应变 $\bm{\varepsilon} = [\varepsilon_{xx}, \varepsilon_{yy}, \gamma_{xy}]^T$ 与位移关系为:
\[
\bm{\varepsilon} = \mathbf{L} \mathbf{u}, \quad \mathbf{L} = \begin{bmatrix} \frac{\partial}{\partial x} & 0 \\ 0 & \frac{\partial}{\partial y} \\ \frac{\partial}{\partial y} & \frac{\partial}{\partial x} \end{bmatrix}
\]
代入位移插值:
\[
\bm{\varepsilon} = \mathbf{L} \mathbf{N} \mathbf{d}^e = \mathbf{B} \mathbf{d}^e
\]
$\mathbf{B}$ 为应变-位移矩阵,对CST单元为常数矩阵:
\[
\mathbf{B} = \frac{1}{2A} \begin{bmatrix} 
b_1 & 0 & b_2 & 0 & b_3 & 0 \\
0 & c_1 & 0 & c_2 & 0 & c_3 \\
c_1 & b_1 & c_2 & b_2 & c_3 & b_3 
\end{bmatrix}
\]
其中:
\begin{itemize}
  \item $A$ 为三角形面积,
  \item $b_i = y_j - y_k$, $c_i = x_k - x_j$($i,j,k$ 循环排列)。
\end{itemize}

\subsection*{3. 应力-应变关系}
线弹性材料本构关系(平面应力或平面应变):
\[
\bm{\sigma} = \mathbf{D} \bm{\varepsilon} = \mathbf{D} \mathbf{B} \mathbf{d}^e
\]
$\mathbf{D}$ 为弹性矩阵,例如平面应力时:
\[
\mathbf{D} = \frac{E}{1-\nu^2} \begin{bmatrix} 
1 & \nu & 0 \\ 
\nu & 1 & 0 \\ 
0 & 0 & \frac{1-\nu}{2} 
\end{bmatrix}
\]

\subsection*{4. 刚度矩阵推导}
通过虚功原理或最小势能原理,单元刚度矩阵为:
\[
\mathbf{K}^e = \int_{\Omega_e} \mathbf{B}^T \mathbf{D} \mathbf{B} \, d\Omega
\]
对于CST单元,$\mathbf{B}$ 和 $\mathbf{D}$ 为常数,且厚度 $t$ 均匀时:
\[
\mathbf{K}^e = t A \, \mathbf{B}^T \mathbf{D} \mathbf{B}
\]

\subsection*{5. 组装与求解}
将各单元刚度矩阵组装成全局刚度矩阵 $\mathbf{K}$,并求解平衡方程:
\[
\mathbf{K} \mathbf{d} = \mathbf{F}
\]
其中 $\mathbf{F}$ 为节点载荷向量,$\mathbf{d}$ 为全局位移向量。

\section*{使用}

\begin{enumerate}
\item 使用$E$ 是弹性模量,$\nu$ 是泊松比计算$\mathbf{D}$矩阵
\item 根据坐标计算出 $A, \mathbf{B}$
\item 根据初始条件和设定计算出 $\mathbf{F}$
\item 联立矩阵求解出 $\mathbf{d}$
\end{enumerate}

\end{document}
