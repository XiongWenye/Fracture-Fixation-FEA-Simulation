\documentclass{article}



\usepackage{arxiv}

\usepackage[utf8]{inputenc} % allow utf-8 input
\usepackage[T1]{fontenc}    % use 8-bit T1 fonts
\usepackage{hyperref}       % hyperlinks
\usepackage{url}            % simple URL typesetting
\usepackage{booktabs}       % professional-quality tables
\usepackage{amsfonts}       % blackboard math symbols
\usepackage{nicefrac}       % compact symbols for 1/2, etc.
\usepackage{microtype}      % microtypography
\usepackage{lipsum}		% Can be removed after putting your text content
\usepackage{graphicx}
\usepackage{natbib}
\usepackage{doi}
\usepackage{amsmath}
\usepackage{bm}            % bold math symbols


\title{Project Report: Fracture Fixation FEA Simulation}

%\date{September 9, 1985}	% Here you can change the date presented in the paper title
%\date{} 					% Or removing it

\author{
  Wenye Xiong \\
  2023533141 \\
  \texttt{xiongwy2023@shanghaitech.edu.cn}
  \And
  Renyi Yang \\
  2023533030 \\
  \texttt{yangry2023@shanghaitech.edu.cn}
  \AND
  Zijian Li \\
  2023533185 \\
  \texttt{lizj2023@shanghaitech.edu.cn}
	%% \AND
	%% Coauthor \\
	%% Affiliation \\
	%% Address \\
	%% \texttt{email} \\
	%% \And
	%% Coauthor \\
	%% Affiliation \\
	%% Address \\
	%% \texttt{email} \\
	%% \And
	%% Coauthor \\
	%% Affiliation \\
	%% Address \\
	%% \texttt{email} \\
}

% Uncomment to remove the date
%\date{}

% Uncomment to override  the `A preprint' in the header
\renewcommand{\headeright}{Technical Report}
\renewcommand{\undertitle}{Technical Report}
\renewcommand{\shorttitle}{\textit{arXiv} Template}

%%% Add PDF metadata to help others organize their library
%%% Once the PDF is generated, you can check the metadata with
%%% $ pdfinfo template.pdf
\hypersetup{
pdftitle={A template for the arxiv style},
pdfsubject={q-bio.NC, q-bio.QM},
pdfauthor={David S.~Hippocampus, Elias D.~Striatum},
pdfkeywords={First keyword, Second keyword, More},
}

\begin{document}
\maketitle

\begin{abstract}
  This report details the implementation and evaluation of a finite element analysis simulation for fracture fixation. We present a comprehensive study examining the mechanical behavior of bone-implant systems under various loading conditions. Our methodology incorporates detailed geometric modeling, material property characterization, and boundary condition specifications to accurately simulate the biomechanical environment.
\end{abstract}


% keywords can be removed
% TODO: keywords
\keywords{FEA \and Fracture Fixation}

\section{Introduction}

In fracture repair processes, as splints or internal fixators gradually reduce load-bearing, the fracture region begins to regain stress-bearing capacity. This project simulates how decreasing splint stiffness (or increasing fracture surface bonding strength) in a "bone+splint" system affects stress redistribution during healing, helping to understand the biomechanical phenomenon of "stress transfer".

\section{FEA Modeling Stress Redistribution During Fracture Healing}

\subsection{Problem Definition}
\begin{itemize}
  \item \textbf{Type}: Non-homogeneous elastic body under load with time-varying material properties (elastic modulus)
  \item \textbf{Model}: 2D "bone+splint" system with either:
        \begin{itemize}
          \item Gradually recovering fracture surface stiffness, or
          \item Progressively decreasing splint stiffness
        \end{itemize}
  \item \textbf{Features}: Simulation of stress transfer from splint to bone tissue
\end{itemize}
\subsection{Mathematical Model}
\subsubsection{Mechanical Framework}
Linear elasticity theory with time-dependent parameters:
\[
  \bm{\sigma}^{(t)} = \mathbf{D}^{(t)}\bm{\varepsilon}
\]
where $\mathbf{D}^{(t)}$ evolves with healing progression.
\subsubsection{Material Evolution}
\begin{align*}
  \text{Bone tissue} & : E_{bone} \approx 17 \text{ GPa (constant)}        \\
  \text{Splint}      & : E_{splint}^{(t)} \text{ (may decrease)}           \\
  \text{Callus}      & : E_{callus}^{(t)} \text{ (increases with healing)}
\end{align*}
\subsubsection{Fracture Interface}
Bonding coefficient $\beta(t)$ represents healing progress:
\[
  \beta(t) \in [0,1] \quad \text{(0 = unhealed, 1 = fully bonded)}
\]
\subsection{Finite Element Implementation}
\subsubsection{CST Element Formulation}
Displacement field for triangular elements:
\[
  \mathbf{u} = \begin{bmatrix} u(x,y) \\ v(x,y) \end{bmatrix} = \mathbf{N}\mathbf{d}^e
\]
Strain-displacement relationship:
\[
  \bm{\varepsilon} = \mathbf{B}\mathbf{d}^e
\]
where $\mathbf{B}$ is the strain-displacement matrix.
\subsubsection{Time-Stepping Algorithm}
\begin{enumerate}
  \item Initialize material properties $E_{callus}(t_0)$, $E_{splint}(t_0)$
  \item Compute element stiffness matrices:
        \[
          \mathbf{K}^{e(t)} = tA\mathbf{B}^T\mathbf{D}^{(t)}\mathbf{B}
        \]
  \item Assemble global system:
        \[
          \mathbf{K}^{(t)}\mathbf{d}^{(t)} = \mathbf{F}^{(t)}
        \]
  \item Update material properties for next time step
\end{enumerate}
\subsection{Expected Outcomes}
\begin{itemize}
  \item Quantification of stress shielding effects
  \item Optimal load transfer protocols
  \item Time-dependent stress distribution visualizations
  \item Correlation between mechanical environment and healing rates
\end{itemize}

\subsection{Results}

\subsection{Discussion}

\section{Conclusion}

\citep{kour2014real} % test_citations


%%%%%%%%%%%%%%%%%%%%%%%%%%%%%%%%%%%%%%%%%%%%%%%%%%%%%%%%%%%%%%%%%%%%%%%%%%%%%%%%%%
%%%%%%%%%%%%%%%%%%%%%%%%%%%%%%%%%%%%%%%%%%%%%%%%%%%%%%%%%%%%%%%%%%%%%%%%%%%%%%%%%%

% \section{Introduction}
% \lipsum[2]
% \lipsum[3]


% \section{Headings: first level}
% \label{sec:headings}

% \lipsum[4] See Section \ref{sec:headings}.

% \subsection{Headings: second level}
% \lipsum[5]
% \begin{equation}
% 	\xi _{ij}(t)=P(x_{t}=i,x_{t+1}=j|y,v,w;\theta)= {\frac {\alpha _{i}(t)a^{w_t}_{ij}\beta _{j}(t+1)b^{v_{t+1}}_{j}(y_{t+1})}{\sum _{i=1}^{N} \sum _{j=1}^{N} \alpha _{i}(t)a^{w_t}_{ij}\beta _{j}(t+1)b^{v_{t+1}}_{j}(y_{t+1})}}
% \end{equation}

% \subsubsection{Headings: third level}
% \lipsum[6]

% \paragraph{Paragraph}
% \lipsum[7]



% \section{Examples of citations, figures, tables, references}
% \label{sec:others}

% \subsection{Citations}
% Citations use \verb+natbib+. The documentation may be found at
% \begin{center}
% 	\url{http://mirrors.ctan.org/macros/latex/contrib/natbib/natnotes.pdf}
% \end{center}

% Here is an example usage of the two main commands (\verb+citet+ and \verb+citep+): Some people thought a thing \citep{kour2014real, hadash2018estimate} but other people thought something else \citep{kour2014fast}. Many people have speculated that if we knew exactly why \citet{kour2014fast} thought this\dots

% \subsection{Figures}
% \lipsum[10]
% See Figure \ref{fig:fig1}. Here is how you add footnotes. \footnote{Sample of the first footnote.}
% \lipsum[11]

% \begin{figure}
% 	\centering
% 	\fbox{\rule[-.5cm]{4cm}{4cm} \rule[-.5cm]{4cm}{0cm}}
% 	\caption{Sample figure caption.}
% 	\label{fig:fig1}
% \end{figure}

% \subsection{Tables}
% See awesome Table~\ref{tab:table}.

% The documentation for \verb+booktabs+ (`Publication quality tables in LaTeX') is available from:
% \begin{center}
% 	\url{https://www.ctan.org/pkg/booktabs}
% \end{center}


% \begin{table}
% 	\caption{Sample table title}
% 	\centering
% 	\begin{tabular}{lll}
% 		\toprule
% 		\multicolumn{2}{c}{Part}                   \\
% 		\cmidrule(r){1-2}
% 		Name     & Description     & Size ($\mu$m) \\
% 		\midrule
% 		Dendrite & Input terminal  & $\sim$100     \\
% 		Axon     & Output terminal & $\sim$10      \\
% 		Soma     & Cell body       & up to $10^6$  \\
% 		\bottomrule
% 	\end{tabular}
% 	\label{tab:table}
% \end{table}

% \subsection{Lists}
% \begin{itemize}
% 	\item Lorem ipsum dolor sit amet
% 	\item consectetur adipiscing elit.
% 	\item Aliquam dignissim blandit est, in dictum tortor gravida eget. In ac rutrum magna.
% \end{itemize}


\bibliographystyle{unsrtnat}
\bibliography{references}  %%% Uncomment this line and comment out the ``thebibliography'' section below to use the external .bib file (using bibtex) .


%%% Uncomment this section and comment out the \bibliography{references} line above to use inline references.
% \begin{thebibliography}{1}

% 	\bibitem{kour2014real}
% 	George Kour and Raid Saabne.
% 	\newblock Real-time segmentation of on-line handwritten arabic script.
% 	\newblock In {\em Frontiers in Handwriting Recognition (ICFHR), 2014 14th
% 			International Conference on}, pages 417--422. IEEE, 2014.

% 	\bibitem{kour2014fast}
% 	George Kour and Raid Saabne.
% 	\newblock Fast classification of handwritten on-line arabic characters.
% 	\newblock In {\em Soft Computing and Pattern Recognition (SoCPaR), 2014 6th
% 			International Conference of}, pages 312--318. IEEE, 2014.

% 	\bibitem{hadash2018estimate}
% 	Guy Hadash, Einat Kermany, Boaz Carmeli, Ofer Lavi, George Kour, and Alon
% 	Jacovi.
% 	\newblock Estimate and replace: A novel approach to integrating deep neural
% 	networks with existing applications.
% 	\newblock {\em arXiv preprint arXiv:1804.09028}, 2018.

% \end{thebibliography}


\end{document}
